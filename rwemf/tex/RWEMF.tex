\documentclass[conference]{IEEEtran}
\ifCLASSINFOpdf
  % \usepackage[pdftex]{graphicx}
  % declare the path(s) where your graphic files are
  % \graphicspath{{../pdf/}{../jpeg/}}
  % and their extensions so you won't have to specify these with
  % every instance of \includegraphics
  % \DeclareGraphicsExtensions{.pdf,.jpeg,.png}
\else
  % or other class option (dvipsone, dvipdf, if not using dvips). graphicx
  % will default to the driver specified in the system graphics.cfg if no
  % driver is specified.
  % \usepackage[dvips]{graphicx}
  % declare the path(s) where your graphic files are
  % \graphicspath{{../eps/}}
  % and their extensions so you won't have to specify these with
  % every instance of \includegraphics
  % \DeclareGraphicsExtensions{.eps}
\fi
\hyphenation{op-tical net-works semi-conduc-tor}

\begin{document}

\title{A hybrid random-walk based web service recommendation enhanced by matrix factorization}

\author{
  \IEEEauthorblockN{Lin Jian}
  \IEEEauthorblockA{School of Physics and \\Electronic Information Engineering, \\WenZhou University\\
  Email: neolinjian@gmail.com}
  \and
  \IEEEauthorblockN{Homer Simpson}
  \IEEEauthorblockA{Twentieth Century Fox\\Springfield, USA\\
  Email: homer@thesimpsons.com}
}

\maketitle

\begin{abstract}
Recently, the Qos(Qaulity of Serivce) of Web Service that includes response time, thoughout put and so on that needs more accuracy prediction. For many web service  callers, choosing the appropriate service in right time should be more significant events. So the web serivce recommendation is right to be the choice. The collaborative filtering is major approach to predict the Qos of more web service through the observed data. But the sparse density of data need new technology to enhance the accuracy of prediction. And the matrix factorization is aslo the common measure to solve the prediction. In this paper, we propose the new hybrid approach that conbined the predictions with random-walk based and matrix facotrizations. Comprehensive experiments on the QoS data set of real-world web service, that our approach achieve the more accuracy predictions.
\end{abstract}

\begin{IEEEkeywords}
  random-work, web service recommendation, matrix factorization
\end{IEEEkeywords}

\IEEEpeerreviewmaketitle


\section{Introduction}
Overview the past five years.
\par Web Service predictions technology developing fastly.
\par The CF(Collaborative Filtering)-based have been widely used.
\par The MF(Matrix Facotrization) has also been chosen for its accuracy.
\par The random-walk that based on RankPage alike measures (the hidden Markov chain theory) to get more appropriate neighbors ranking with the transition matrix.
\par The contributions we made as following:
1. We conbine the cf with mf.
2. We try to examine the elements which affects the accuracy of experiments.
3. We conduct the experiments on real-world datasets. achiving the most accuracy MAE.
4. Right this is test

\section{Initition of sparse density data}
\subsection{A. What  }
Subsection text here.
\subsection{B. How to  }
Subsection text here.
\subsection{C. Is there extreme rate of data mining  }
Subsection text here.


\section{Related Work}
The conclusion goes here.
\subsection{A. Collaborative Filtering on user }
Subsection text here.

\subsubsection{B. Random-walk neighbors ranking}
Subsubsection text here.

\subsubsection{C. Matrix facotorization }
Subsubsection text here.

\section{Hybrid appoach with RW and MF}
Ssdfsdfsdfsdfsdf

\section{Conclusion}
 The conclusion can be summarized by this.

% use section* for acknowledgment
\section*{Acknowledgment}


The authors would like to thank...


\begin{thebibliography}{1}

\bibitem{IEEEhowto:kopka}
H.~Kopka and P.~W. Daly, \emph{A Guide to \LaTeX}, 3rd~ed.\hskip 1em plus
  0.5em minus 0.4em\relax Harlow, England: Addison-Wesley, 1999.

\end{thebibliography}

\end{document}


